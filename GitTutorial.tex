\documentclass{article}

% Language set to English
\usepackage[english]{babel}

% Set page size and margins
\usepackage[letterpaper,top=2cm,bottom=2cm,left=3cm,right=3cm,marginparwidth=1.75cm]{geometry}

% Useful packages
\usepackage{amsmath}
\usepackage{graphicx}
\usepackage[colorlinks=true, allcolors=blue]{hyperref}

\title{Source Control for Writers}
\author{Jennifer Manzella}

\begin{document}
\maketitle

\begin{abstract}
This document covers Git Workflow
\end{abstract}

\section{Introduction}
    In this tutorial, I will cover local git configuration and basic workflow.
	If you don't already have an account, please begin by creating one through
	the  \href{https://www.github.com}{github website}. By the end of the
	tutorial, you will understand the following:
    \subsection{Local Configuration}
    \begin{itemize}
        \item Configuring Git on Your Local System
        \item Generating a Public Key
        \item Adding Public Key to Github Account
        \item Cloning Your Forked Repository
        \item Setting Upstream Remotes
        \end{itemize}
    \subsection{Git Workflow}
    \begin{itemize}
        \item Understanding Workflow
        \item Creating Your Working Branch
        \item Working with Branches
        \item Pushing Local Changes Upstream
        \item Generating Diff Files
        \item Examples
        \item Secure Copy
        \item Squashing and Formatting Commit
        \item Creating a PR
        \item Useful Git Commands
    \end{itemize}

\section{Git Install}
Start by installing Git on your local computer if you haven't already.
The github installation guide is  
 \href{https://github.com/git-guides/install-git}{here}.
 
 \subsection{Initialize Environment}
After installation completes, you can set up your local environment.
Open the terminal by typing "cmd" in the search field. If Git installed
successfully, you should be able to



\end{document}